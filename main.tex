\documentclass{article}
\usepackage[utf8]{inputenc}
\usepackage[russian]{babel}

\title{Анализ плагиата в практических курсах по программированию}
\author{Андрей Цибин и Евгений Ефимчик}
\date{Октябрь 2018}

\usepackage{natbib}
\usepackage{graphicx}
\usepackage{url}

\begin{document}

\maketitle

\begin{abstract}
Задачи поиска и анализа плагиата применительно к практическим курсам по программированию становятся с каждым годом все более важными. Все больше дисциплин, связанных с Computer Science, начинают включать в себя задания по программированию, причем их сложность может значительно варьироваться. Поэтому, искать плагиат в заданиях, связанных с базовыми алгоритмами или конструкциями некоторого языка программирования, будет тупиковой затеей, поскольку исходный код ожидаемо будет сильно совпадать, но при этом вероятность того, что такой код действительно является заимствованным, будет недостоверно низкой. Тем не менее, если сложность задачи, поставленной перед обучающимися, предоставляет значительное пространство возможных решений, то поиск случаев заимствования становится резонным.

Важно понимать, что практическую пользу приносит не сам поиск плагиата, сколько анализ результатов этого поиска. Правильная интерпретации полученных данных является непосредственным фактором, влияющим на то, какие работы необходимо классифицировать, как заимствованные, а какие - как уникальные.
\end{abstract}

\section{Введение}

В 2017 году на кафедре Компьютерных Образовательных Технологий Университета ИТМО появилась необходимость в создании автоматизированной системы сдачи и проверки практических заданий по программированию. Изначально, задача включала в себя автоматизацию процессов проверки тестов и оценки качества кода, а также процесса анализа плагиата решений обучающихся.

За промежуток в чуть более полугода система\footnote{Образовательная платформа Flaxo: \url{https://github.com/tcibinan/flaxo}} была разработана и содержала в себе, помимо прочего, возможность анализа плагиата решений обучающихся, находящихся в различных репозиториях на платформе GitHub\footnote{Вендор Git: \url{https://github.com}}. Анализ плагиата с помощью платформы включал в себе возможность простейшей визуализации найденных случаев заимствования исходного кода.

Процесс построения и внедрения системы в отношении под-системы анализа плагиата приведен далее в настоящей статье.

\section{Заключение}
``I always thought something was fundamentally wrong with the universe'' \citep{adams1995hitchhiker}

\bibliographystyle{plain}
\bibliography{references}
\end{document}
