\documentclass{article}
\usepackage[utf8]{inputenc}
\usepackage[russian]{babel}

\title{Анализ плагиата в практических курсах по программированию}
\author{Андрей Цибин, Евгений Ефимчик}
\date{Октябрь 2018}

\usepackage{natbib}
\usepackage{graphicx}
\usepackage{url}

\begin{document}

\maketitle

\begin{abstract}

Проблема недобросовестного заимствования в академической среде по-прежнему остается актуальной проблемой. Особо остро она проявляется в среде преддипломного образования. Причинами этому служат многие факторы, в частности, отсутствие преподавания дисциплины научной этики. Недобросовестное заимствование, или плагиат, встречается сегодня в различных формах академической активности, начиная от семестровых работ студентов и заканчивая диссертациями ученых. Отдельной проблемой является недобросовестное заимствование в работах обучающихся учебных заведений, которые они выполняют в рамках практических курсов по программированию. Как и в случае с текстом, выявлять плагиат в ручную является возможным только в самых небольших подвыборках данных. К счатью, на сегодняшний день существуют довольное большое количество систем, позволяющих выявлять сходство программного кода автоматизированно. Более того, существуют средства, позволяющие аггрегировать результаты поиска плагиата несколькими различными системами, что также увеличивает вероятность обнаружения случаев недобросовестного заимствования. При этом, данные средства по-прежнему не так распространены, по крайней мере, в отечественных образовательных учреждениях. Вероятно, это вызвано высокой сложностью интерпретации результатов систем анализа плагиата исходного кода.

В настоящей статье, продемонстрирован процесс проектирования и внедрения системы анализа плагиата в образовательный процесс, а также описан интерактивный инструмент графовой визуализации плагиата исходного кода.

\end{abstract}

\section{Введение}

Задача выявления плагиата применительно к практическим курсам по программированию становится с каждым годом все более актуальной. Именно по этой причине, в 2017 году на кафедре Компьютерных Образовательных Технологий Университета ИТМО с целью повышения качества автоматизации процесса обучения появилась необходимость в создании автоматизированной системы сдачи и проверки практических заданий по программированию. Изначально, поставленная задача включала в себя автоматизацию процессов проверки работ обучающихся по критериям прохождения ими тестов, оценки качества кода и анализа результатов выявления плагиата.

Система\footnote{Образовательная платформа Flaxo. \url{https://github.com/tcibinan/flaxo}} была разработана в рамках бакалаврской выпускной квалификационной работы \citep{flaxoThesis}, и с начала учебного года 2018-2019 постепенно внедряется в образовательный процесс кафедры.

В следующих разделах описаны этапы проектирования и создания модуля анализа плагиата указанной выше системы.

\section{Инфраструктура}

Концепция оригинальной образовательной платформы включала хранение всех заданий и решений обучающихся в репозиториях системы контроля версий. В частности, в текущем варианте системы существует поддержка системы версионирования Git\footnote{Распределенная система версионирования Git. \url{https://git-scm.com/}}, а в качестве вендора системы - платформа GitHub\footnote{Платформа разработки GitHub. \url{https://github.com/}}, которая на сегодняшний день является одной из самых популярных среди open-source проектов.

Для того чтобы провести анализ существующих решений, необходимо сначала собрать весь исходный код: файлы, написанные преподавателем, которые называются \textit{базовыми}, а также все изменненные или созданные обучающимися. Учитывая, что репозитории с заданиями и решениями помимо исходного кода могут содержать множество других файлов, например, скрипты системы сборки проекта или документацию, необходимо перед выгрузкой заданий и решений фильтровать содержания репозиториев по расширениям файлов. Также загружаемые файлы должны быть сгруппированы по обучающемуся, который их создал или изменил, чтобы дальнейший анализ мог различать автора того или иного фрагмента кода. В качестве хранилища файлов, можно использовать файловую систему, поскольку количество и размер загружаемых файлов могут быть значительными.

\section{Инструмент анализа}

Одна из основных задач, возникающая при попытке внедрения системы анализа плагиата это, безусловно, подбор существующих инструментов выявления сходства программного кода. На сегодняший день, существует значительное число различных инструментов для этих целей \citep{plagiarismToolsSurvey}. Они различаются механизмом анализа плагиата, форматами входных и выходных данных, скоростью обработки программного кода, а также количеством поддерживаемых языков. Помимо прочего, на основе проанализированной литературы, существует как минимум один инструмент, который унифицирует результаты нескольких других инструментов анализа плагиата, предоставляя более точное значение вероятности плагиата для каждого из решений \citep{unifiedPlagiarismDetectionTool}.

При разработке настоящего модуля анализа плагиата был выбран инструмент Moss \citep{mossOriginalPaper}, обладающий рядом характеристик, описанных далее в данном разделе.

Moss является веб-сервисом анализа плагиата, поддерживающим анализ плагиата на 26 языках программирования. Сервис предоставляет Perl-скрипт в качестве клиентского приложения, который позволяет загружать файлы с исходным кодом и инициализировать анализ плагиата. Наряду с обычными файлами, Moss использует упомянутое выше понятие базовых файлов. Под этим термином подразумеваются файлы, созданные преподавателем, строки которых должны быть проигнорированы при выявлении заимствованных фрагментов исходного кода.

Как было сказано выше, Moss принимает на вход набор файлов для анализа, а на выходе генерирует набор HTML-страниц, которые содержат подробные отчеты по каждой паре решений, содержащих заимствованные коды. Отчет включает в себя количество и процент общих строчек всех файлов, а также таблицы с исходным кодом решений, включающие цветовое обозначение общих фрагментов кода. Поскольку формат HTML не пригоден для автоматизированной обработки, каждую сгенерированную страницу необходимо самостоятельно парсить. А учитывая факт, что сервис хранит результаты анализа только 14 суток, после чего удаляет их, необходимо полностью выгружать данные до истечения этого срока.

\bibliographystyle{plain}
\bibliography{references}
\end{document}
